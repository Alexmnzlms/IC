\chapter{Sistema Experto}
El sistema experto presentado sirve para asesorar a un alumno a la hora de escoger que mención realizar o que asignaturas matricularse.\\
Esta divido en dos SBC:
\begin{itemize}
   \item Un SBC se encarga de aconsejar a un estudiante que mención de la carrera de Ingeniería Informática (de la UGR) debería escoger. Para esto realiza una serie de cuestiones al usuario de las cuales obtiene la información necesaria para realizar un consejo. Implementa un razonamiento por defecto y se trata de una mejora del SBC presentado en la práctica 1.
   \item Otro SBC se encarga de seleccionar en que asignaturas debería matricularse un estudiante. Para esto, necesita que se le proporcione un número de créditos y las asignaturas sobre las que se tiene dudas. Una vez sabe esto, realiza una serie de preguntas al usuario a través de las cuales razona que asignaturas debe recomendar. Implementa un razonamiento basado en factores de certeza.
\end{itemize}

El funcionamiento detallado de los módulos de explicará en los siguientes apartados. También se aporta una guía de uso del sistema.

\chapter{Proceso seguido}
\section{Desarrollo de la base de conocimiento}
Para el desarrollo de la base del conocimiento se ha utilizado la propia experiencia a la hora de diseñar el sistema:\\
\\
Para el SBC que recomienda en que rama debe matricularse un estudiante, se ha decidido establecer una serie de elementos que definen conceptos que pueden asociarse a cada rama. Un mismo elemento puede asociarse a varias ramas. Concretamente se definen los siguientes elementos:
\begin{itemize}
   \item Matemáticas: Se asocia a la rama de CSI.
   \item Programar: Se asocia a la rama de CSI e IS.
   \item Bases de datos: Se asocia a la rama de SI.
   \item Hardware: Se asocia a la rama de IC.
   \item Docencia: Se asocia a la rama de CSI y SI.
   \item Web: Se asocia a la rama IS y TI.
   \item Sistemas: Se asocia a la rama de IC y SI.
   \item Videojuegos: Se asocia a la rama de IS.
   \item Robótica: Se asocia a la rama de CSI e IC.
   \item Red: Se asocia a la rama de TI.
\end{itemize}
Este SBC implementa un razonamiento por defecto para recomendar una rama. Por defecto se recomiendan todas las ramas dando un motivo por defecto. Se le pregunta al usuario si le gustan cada uno de los elementos anteriores. Si especifica que no le gusta un elemento asociado a una rama, no recomendará esa rama (eliminando la recomendación por defecto), pero siempre que le guste un elemento asociado a una rama, la recomendará dando los motivos por los que realiza la recomendación.\\
\\
Para el SBC que recomienda que asignaturas, se establecen una serie de características que pueden asociarse a asignaturas concretas:
\begin{itemize}
   \item Primero: Especifica las asignaturas que son de primero.
   \item Hardware: Especifica las asignaturas que centran su contenido en torno a temas hardware.
   \item Software: Especifica las asignaturas que se centran en desarrollo software.
   \item Matemáticas: Especifica las asignaturas que tienen una gran carga matemática.
   \item Programación: Especifica las asignaturas que tienen una gran carga de programación.
   \item Teoría: Especifica las asignaturas que se centran más en la parte teórica que en la practica.
   \item Practica: Especifica las asignaturas que centran más en la parte practica que en la teórica.
\end{itemize}

Se ha decidido limitar el conjunto de asignaturas a las asignaturas de primero y segundo curso, dado que son las asignaturas sobre las que más conocimiento posee el experto, dado que las ha cursado todas.
Este SBC implementa un razonamiento basado en factores de certeza. Se establece a 0 la certeza de recomendar las asignaturas propuestas. Mediante una serie de preguntas, se establecen los gustos del usuario, aumentando el factor de certeza asociado a las asignaturas que especifican los gustos. A la hora de recomendar, se recomendara la asignatura si la certeza supera el 50\%.

\section{Validación y verificación}
Para validar el sistema y comprobar su correcto funcionamiento, se han realizado una serie de pruebas.
\\\\
El sistema de recomendación de ramas, funciona correctamente, dado que es capaz de proporcionar una recomendación por defecto. También es capaz de no recomendar una rama si ninguna de las características que la definen le gusta al usuario mientras que si alguna le gusta, recomienda la rama dando las explicaciones correctas.
\\\\
El sistema de recomendación de asignaturas también funciona correctamente. Dado una cantidad de créditos es capaz de calcular cuantas asignaturas necesita y se las solicita al usuario. También es capaz de aceptar más asignaturas si el usuario así lo quiere. Razona correctamente el factor de certeza para recomendar una asignatura según los gustos del usuario. Finalmente es capaz de recomendar tantas asignaturas como créditos se le hayan proporcionado, además de dar los motivos por los cuales los hace la recomendación.

\chapter{Descripción del sistema}
Los sistemas funcionan de la siguiente manera:
\begin{itemize}
   \item Recomendador de rama: El sistemas realiza una serie de preguntas y en base a las respuestas que de el usuario, mejor es la recomendación que aporta, pudiendo obtener una recomendación con información incompleta.
   \item Recomendador de asignatura: El sistema solicita un número de créditos y calcula a cuantas asignaturas corresponde. Después pregunta al usuario por las asignaturas sobre las que quiere ser aconsejado. A continuación pregunta por los gustos del usuario y razona el grado de recomendación de cada asignatura. Finalmente recomienda tantas asignaturas como antes había calculado en base a los créditos proporcionados.
\end{itemize}
\section{Recomendador de rama}
\subsection{Variables de entrada}
Las variables de entrada de este SBC son los gustos del usuario. Estos se representan mediante el hecho (gusta ?elemento ?(si/no))
\subsection{Variables de salida}
Las variables de salida son si una rama se aconseja o no, o si esta se aconseja por defecto. Esto se representa mediante los hechos (consejo ?rama ?(si/no/por\_defecto)). También es una variable de salida la explicación de la recomendación. Esto se representa mediante los hechos (expl\_consejo ?rama ?explicacion)
\subsection{Conocimiento global del sistema}
Para este SBC se deben establecer las ramas que existen mediante el hecho (rama ?rama), también la relación entre un elemento y la rama a la que se lo asocia (relación ?elemento ?rama). También debemos asertar la explicación asociada a cada elemento mediante el hecho (explicación ?elemento ?explicacion).\\
Por último establecemos vacía la recomendación de cada rama mediante (expl\_consejo ?rama "").\\
También establecemos por defecto el módulo inicial que se ejecutará mediante (modulo ?modulo).
\subsection{Módulos}
\subsubsection{Estructura}
Este SBC cuenta con un total de 6 módulos que se ejecutan de manera secuencial. Existen los módulos de: razonamiento por defecto, preguntas, razonamiento, consejo, motivos y explicación.

\subsubsection{Descripción, hechos y reglas}
\begin{itemize}
   \item Módulo razonamiento\_por\_defecto: Este módulo establece el hecho (consejo ?rama por\_defecto) para establecer el razonamiento por defecto del SBC.
   \\\\
   \textbf{Reglas que utiliza}: Consejo\_por\_defecto es la regla encargada de establecer el consejo por defecto y Comenzar\_preguntas se encarga de pasar al siguiente módulo.

   \item Módulo preguntas: Este módulo simplemente trata de las reglas necesarias para realizar preguntas al usuario y obtener la información. Cada pregunta aserta un hecho (gusta ?elemento ?(si/no)). También contiene las reglas necesarias para poder cortar el flujo de preguntas en cualquier momento.
   \\\\
   \textbf{Reglas que utiliza}: Las reglas Primera\_pregunta hasta Decima\_pregunta funcionan de la misma manera, simplemente preguntan al usuario por su gusto por un elemento concreto y asertan el hecho (gusta ?elemento (read)) con la respuesta del usuario. También se utiliza el hecho (terminar) para controlar que se ejecute la regla Preguntar\_Final que simplemente pregunta si quiere terminar de responder preguntas. Si la respuesta es si, se aserta el hecho (terminado si) que impide que se realicen más preguntas. La regla Final\_preguntas simplemente sirve para pasar al siguiente módulo.

   \item Módulo razonamiento: Este módulo contiene las reglas necesarias para razonar en base a los hechos (gusta ?elemento ?(si/no)) y (relación ?elemento ?rama), el hecho de que al usuario le guste o no un elemento especifico de cada rama mediante el hecho (gusta\_rama ?elemento ?rama ?(si/no)).
   \\\\
   \textbf{Reglas que utiliza}: Razonar\_gusto\_rama sirve para realizar el razonamiento previamente comentado y Fin\_razonamiento sirve para cambiar de módulo.

   \item Módulo consejo: Este módulo es el encargado de razonar si una rama se aconseja o no. Esto se establece mediante el hecho (consejo ?rama ?(si/no)).
   \\\\
   \textbf{Reglas que utiliza}: Las reglas Razonar\_consejo y Razonar\_consejo\_si se encargan de razonar si se debe o no recomendar una rama. Cuando existe el hecho (gusta ?elemento ?rama ?si), se activa Razonar\_consejo\_si y se recomienda esta rama. Razonar\_consejo es una regla que sirve para eliminar el consejo por defecto e inferir que no se debe recomendar una rama. Fin\_consejos sirve para pasar al siguiente módulo.

   \item Módulo motivos: Este módulo es capaz de inferir los motivos por los cuales se aconseja una rama.
   \\\\
   \textbf{Reglas que utiliza}: Motivos\_consejo\_por\_defecto establece los motivos de la recomendación por defecto y Motivos\_consejo establece los motivos por los cuales se recomienda una rama según las respuestas del usuario. Fin\_motivos sirve para pasar al módulo final.

   \item Módulo explicación: Este módulo simplemente le muestra al usuario las ramas que se recomiendan y el por que de esta recomendación.
   \\\\
   \textbf{Reglas que utiliza}: Expl\_razonada es la regla que simplemente muestra información al usuario.

\end{itemize}

\section{Recomendador de asignaturas}
\subsection{Variables de entrada}
Las variables de entrada de este SBC son el número de créditos que proporciona el usuario, que se traduce al número de asignaturas para las que necesita consejo. Esto se representa con el hecho (n\_asig ?numero). También es una variable de entrada las asignaturas sobre las que el usuario esta interesado. Esto se almacena mediante el hecho (interes ?asignatura). Por último, al igual que en el recomendador de rama, es necesario almacenar los gustos del usuario respecto al tipo de asignatura, esto se almacena mediante (gusta ?elemento ?(si/no)).
\subsection{Variables de salida}
Las variables de salida son en esta caso, los factores de certeza para recomendar cada asignatura. Esto se representa mediante el hecho (FactorCerteza ?asignatura ?certeza). También es una variable de salida el motivo por el cual se realiza la recomendación. Esto se representa con el hecho (expl\_consejo ?asignatura ?explicacion).
\subsection{Conocimiento global del sistema}
Inicialmente se deben establecer las asignaturas mediante el hecho (asignatura ?asignatura) y el nombre de esta mediante (nombre ?asignatura ?nombre). Después se establece que asignaturas son de primero mediante el hecho (primero ?asignatura). También se establecen las relaciones entre las distintas características de cada asignatura por las que se preguntan con las asignaturas a las que corresponden mediante el hecho (relacion ?elemento ?asignatura). Finalmente se establecen las explicaciones de cada características mediante (explicacion ?elemento ?explicacion) y los motivos por los cuales se recomienda cada asignatura (expl\_consejo ?asignatura ""). También se establece un hecho de control (maximo\_necesario) y el módulo inicial (modulo créditos).
\subsection{Módulos}
\subsubsection{Estructura}
Este SBC cuenta con un total de 7 módulos. Estos módulos son: créditos, asignaturas, intereses, preguntas, gusto, certeza y razonar certeza.
\subsubsection{Descripción, hechos y reglas}
\begin{itemize}
   \item Módulo créditos: Este módulo se encarga de solicitar al usuario el número de créditos y calcular las asignaturas correspondientes. También controla que el número de créditos establecido no sea mayor que el máximo posible (30 en este caso, que son los créditos máximos por cuatrimestre).
   \\\\
   \textbf{Reglas que utiliza}: La regla Numero\_de\_asignaturas pregunta por el número de créditos. La regla Creditos simplemente muestra el calculo que se ha hecho y establece un contador para las asignaturas a solicitar. Creditos\_exceso es una regla que rompe el flujo de ejecución mediante el hecho (exceso) si el número de créditos se ha excedido. Contador\_maximo es una regla necesaria para posteriormente obtener el factor de certeza máximo. Fin\_creditos es la regla que permite pasar al siguiente módulo.

   \item Módulo asignaturas: Este módulo se encarga de pedir asignaturas al usuario. Se ha implementado un contador de manera que el usuario tenga que introducir tantas asignaturas como asignaturas se hayan establecido por créditos. Aun así, puede introducir hasta un máximo de 10 asignaturas.
   \\\\
   \textbf{Reglas que utiliza}: Pedir\_asignaturas se encarga de solicitar una asignatura y establece el hecho de control (confirmar ?asignatura) que sirve para que no se establezca intereses por una asignatura que no esta almacenada. Confirmar\_asignatura y Confirmar\_asignatura\_max son las reglas que confirman que la asignatura existe y asertan el hecho (interes ?asignatura) y controlan el contador. Existen dos funciones porque una sirve para establecer (terminar\_asig) y llamar a la regla Terminar\_asig\_max que pregunta si se ha terminado de introducir asignaturas una vez superado el mínimo. Las reglas Confirmar\_asignatura\_no\_valida y Confirmar\_asignatura\_no\_valida\_max sirve  para que si se introduce una asignatura no valida esta no tenga efecto en el sistema. Finalmente Fin\_asignaturas permite pasar al siguiente módulo.

   \item Módulo intereses: Este módulo es sencillo y simplemente se encarga de establecer el factor de certeza de las asignaturas por las que se ha mostrado interés.
   \\\\
   \textbf{Reglas que utiliza}: Intereses establece el hecho (FactorCerteza ?asignatura 0) para cada (interes ?asignatura). Fin\_intereses sirve para pasar al siguiente módulo.

   \item Módulo preguntas: Este módulo funciona de manera muy similar al módulo con el mismo nombre del SBC recomendador de rama. Con la diferencia que ahora existen solo 6 preguntas.

   \item Módulo gusto: Una vez se ha obtenido los gustos del usuario, este módulo se encarga de establecer el hecho (gusta\_tipo ?elemento ?asignatura) que establece una relación entre las asignaturas de interés y los elementos que el usuario ha indicado como de su gusto.
   \\\\
   \textbf{Reglas que utiliza}: La regla Razonar\_gusto se encarga de realizar el procedimiento antes mencionado. Fin\_gusto permite pasar al siguiente módulo.

   \item Módulo certeza: Este módulo es el encargado de razonar el factor de certeza que se debe aplicar a cada asignatura. Para cada elemento se establece un factor de certeza de la siguiente manera: \\$1 - \frac{n\_de\_asignaturas\_relacionadas\_con\_el\_elemento}{n\_total\_de\_asignaturas}$\\ De esta manera se establece que un elemento con menor número de asignaturas asociadas indica que si al usuario le gusta este elemento es muy probable que las asignaturas relacionadas con el le gusten mucho. Si hay muchas asignaturas relacionadas este factor será mas pequeño. Se utiliza la función de combinación que se proporcionó en el ejercicio sobre factores de certeza.
   \\\\
   \textbf{Reglas que utiliza}: Las reglas Certeza\_* se encargan de combinar el factor de certeza actual de una asignatura con el factor asociado a un elemento concreto. También actualizan los motivos de la recomendación en caso de que esta llegue a realizarse. La regla Fin\_certeza permite pasar al siguiente módulo.

   \item Módulo razonar\_certeza: Este módulo se encarga de recomendar tantas asignaturas como asignaturas se hayan establecido a través de créditos. Para esto selecciona la asignatura con un valor de certeza máximo y se lo devuelve al usuario de manera reiterada utilizando un contador.
   \\\\
   \textbf{Reglas que utiliza}:La regla Factor\_maximo se encarga de razonar la asignatura con un factor de certeza máximo. Factores\_certeza\_muy\_seguro Y Factores\_certeza\_seguro se encargan de recomendar al usuario las asignaturas escogidas y mostrar los motivos por los que se realiza la recomendación. Se considera que el SBC esta seguro si el factor es superior al 50\% y esta muy seguro si es superior al 80\%. Si no, se activa la regla Factores\_certeza\_no\_seguro que se encarga de no recomendar una asignatura en el caso de ser necesario.

\end{itemize}


\chapter{Manual de uso}
Finalmente se adjunta un manual de uso:
\begin{itemize}
   \item Lo primero que solicita el programa es que se establezca si queremos ser aconsejados sobre la rama que escoger o sobre las asignaturas.
   \item Si seleccionamos rama, el sistema comenzara a realizar preguntas de si o no y cada vez que respondamos nos preguntara si queremos terminar. Si le decimos que si, el sistema realizará una recomendación con información imperfecta, mientras que si le decimos que no, continuara preguntando hasta que las preguntas se agoten y el sistema devuelve su recomendación final.
   \item Si seleccionamos asig, deberemos hacer lo siguiente:
   \begin{itemize}
      \item Se nos mostrará un resumen de las asignaturas disponibles.
      \item Primero introducir un número de créditos (por ejemplo 18).
      \item Después debemos proporcionar el número de asignaturas necesario (3 en nuestro ejemplo) o proporcionar asignaturas hasta un máximo de 10.
      \item Si llegamos al máximo o decidimos responder si a la pregunta de si hemos terminado, el sistema comenzara a hacernos preguntas y después de cada pregunta nos preguntara si hemos terminado. Si le decimos que si, realizará una recomendación con información incompleta, pero si agotamos las preguntas, finalmente nos recomendara tantas asignaturas como asignaturas se hayan establecido por créditos.
   \end{itemize}
\end{itemize}
