\documentclass{article}
\usepackage[utf8]{inputenc}

\title{Revisión de la documentación sobre Integración de CLISP en C y Python}
\author{Alejandro Manzanares Lemus}

\usepackage{natbib}
\usepackage{graphicx}
\date{}

\begin{document}

\maketitle
\textbf{¿Cómo se instala CLIPSPy?}\\
Se instala a través de pip o compilándolo manualmente
\\\\
\textbf{¿Se pueden evaluar en Python (mediante CLIPSPy) expresiones de CLIPS? ¿Cómo?}\\
Se puede.\\
Utilizando el método eval() de la clase clips.environment.Environment
\\\\
\textbf{Mediante CLIPSPy se pueden utilizar dentro de CLISP funciones de pythom. ¿Cómo se hace?}\\
Con el método define\_function() de la clase clips.environment.Environment
\\\\
\textbf{¿Qué métodos de CLIPSPy asertan y retractan un hecho en CLISP?}\\
Se puede asertar un string con el metodo assert\_string de la clase clips.environment.Environment\\
También con los  assertit() y retract() de la clase clips.facts.Fact().
\\\\
\textbf{¿Cómo se ejecutaría en Python (mediante CLIPSPy) un sistema basado en reglas definido mediante un fichero .clp?}\\
Mediante el método load de la clase clips.environment.Enviroment()  y run de la clase clips.agenda.Agenda()
\\\\
\textbf{Describe brevemente como convertirías un sistema basado en reglas definido mediante un fichero .clp en un fichero ejecutable}\\
Incluyendo clips.h en un fichero en C\\
// creamos el entorno\\
void * entorno = CreateEnvironment();\\
// cargamos el clp\\
EnvLoad(entorno, “fichero.clp”)\\
// lo ejecutamos\\
EnvRun(entorno, -1)\\
Y compilándolo en C con gcc\\
\\\\
\newpage
\textbf{Describe brevemente cómo incluirías en CLISP una función definida en C}\\
Con la función int EnvDefineFunction(environment,functionName,functionType,\\functionPointer,actualFunctionName);
\\\\
\textbf{Describe brevemente cómo incluirías un sistema basado en reglas definido mediante un fichero .clp dentro de tu programa escrito en C.}\\
Incluyendo clips.h en un fichero en C\\
// creamos el entorno\\
void * entorno = CreateEnvironment();\\
// cargamos el clp\\
EnvLoad(entorno, “fichero.clp”)\\
// lo ejecutamos\\
EnvRun(entorno, -1)\\
Y compilándolo en C con gcc\\
\\\\
\textbf{¿Que funciones se utilizan para asertar o retractar un hecho en un sistema basado en reglas embebido en un programa de C?}\\
 EnvAssert y EnvRetract para asertar y rechazar o EnvAssertString para asertar un string como regla
\\\\
\textbf{¿Se pueden ejecutar varios sistemas basados en reglas distintos dentro de un mismo programa de C?}\\
Creando varios entornos:\\
// creamos el entorno\\
void * entorno = CreateEnvironment();\\
void * entorno1 = CreateEnvironment();\\
// cargamos el clp\\
EnvLoad(entorno, “fichero.clp”);\\
EnvLoad(entorno1, “fichero1.clp”);\\
// lo ejecutamos\\
EnvRun(entorno, -1);\\
EnvRun(entorno1, -1);\\

\end{document}
